% phuc
\begin{abstract}
    Trước đây, những nhà nghiên cứu thị trường cho rằng việc dự đoán giá cổ phiếu là việc không thể, vì nó phụ thuộc nhiều vào sự biến động của thị trường. Nhưng ngày nay lại có những đề xuất chứng minh rằng việc sử dụng các mô hình thống kê, thuật toán học máy hay học sâu nếu được thiết kế thích hợp thì việc dữ đoán giá cổ phiếu có thể được thực hiện rất chính xác. Trong nghiên cứu này, chúng em dựa trên các mô hình đã có sẵn để thực hiện dự đoán giá cổ phiếu của 3 ngân hàng lớn tại Việt Nam: Ngân hàng Thương mại cổ phần Ngoại thương Việt Nam (VCB), Ngân hàng Thương mại cố phần Đầu tư và Phát triển Việt Nam (BIDV), Ngân hàng thương mại cổ phần Xuất Nhập khẩu Việt Nam (Eximbank). Để tìm ra các mô hình nào tốt nhất, chúng em lần thực hiện các mô hình sau trên bộ dữ liệu của cả 3 ngân hàng để so sánh: Hồi quy tuyến tính (LR), Đường trung bình động tích hợp tự hồi quy (ARIMA), Mạng thần kinh tái phát (RNN), Đơn vị định kì có kiểm soát (GRU), Bộ nhớ ngắn hạn dài hạn (LSTM), thuật toán siêu học (Meta-Learning), Phân tích mở rộng cơ sở thần kinh (NBeats), Bộ lọc Kalman (Kalman Filter), trung bình động tự hồi quy vector (VARMA), Nội suy phân cấp thần kinh để dự báo chuỗi thời gian (N-HiTS). Ngoài ra, cần đo hiệu quả của các mô hình bằng cách sử dụng các độ đo sau: Lỗi bình phương trung bình gốc (RMSE), Lỗi phần trăm tuyệt đối trung bình (MAPE), Lỗi tuyệt đối trung bình (MAE).
\end{abstract}
    
\begin{keywords}
    Stock price prediction, machine learning, deep learning.
\end{keywords}
% phucs